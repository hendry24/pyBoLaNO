\documentclass[aps,reprint,onecolumn,amsmath,amssymb,showpacs]{revtex4-1}
\usepackage{graphicx,bm} \usepackage{xcolor} \usepackage{floatrow}  \usepackage{ragged2e} \usepackage[paper=a4paper,left=25mm,right=25mm,top=30mm,bottom=25mm]{geometry}
\usepackage{kantlipsum}
\usepackage{hyperref}
\renewcommand{\baselinestretch}{1.3} 

\begin{document}
\pagestyle{empty}

\begin{flushleft}

Dear Editor of Computer Physics Communications (CPC),
\vspace{10px}

Enclosed with this message, we are submitting a paper entitled:
\vspace{5px}

\textbf{\texttt{pyBoLaNO}: A \texttt{Python} symbolic package for normal ordering involving bosonic ladder operators}

\vspace{5px}

by Hendry M. Lim, Donny Dwiputra, M Shoufie Ukhtary, and Ahmad R. T. Nugraha, for publication in CPC.
\vspace{5px}

\end{flushleft} 

\begin{justify}
\vspace{3px}

In this work, we present a Python-based \textbf{symbolic} computation package, which we name as \texttt{pyBoLaNO}, for efficiently handling the normal ordering of bosonic ladder operators.  By automating this process, \texttt{pyBoLaNO} offers researchers a powerful tool to shortcut some tedious analytical calculations in quantum mechanics and related disciplines, especially research activities involving expectation value evolution in the Lindblad master equation framework.

\vspace{3px}

In the manuscript, we discuss the underlying algorithms of \texttt{pyBoLaNO} and demonstrate its practical utility through applications in quantum mechanical problems, particularly in the context of operator algebra, such as for various types of quantum oscillators and resonators, including the dissipative systems and multipartite setups.  The results we present by \texttt{pyBoLaNO}'s calculations underscore the versatility and robustness of the package, which can simplify complex symbolic operations and improve analytical reproducibility.  Moreover, the package integrates well with the Python ecosystem, thus making it an invaluable asset not only for physicists but also for computational scientists in general.

\vspace{3px}

We believe that \texttt{pyBoLaNO} represents a significant contribution to the field of computational physics, as served by CPC.  We appreciate your consideration of our manuscript and look forward to the opportunity of publication in CPC.  To assist in the review process, we would like to suggest the following potential reviewers who (or whose groups) could be potential users of this package:

\begin{enumerate}
    \item Prof. Pawel Blasiak (Institute of Nuclear Physics, Polish Academy of Sciences).  He is an expert in the foundation of quantum mechanics.  We use his formulae to develop the \texttt{pyBoLaNO} package.  We cite his papers as Refs. [25] and [26] in the manuscript. His email is Pawel.blasiak[at]ifj.edu.pl
    \item Dr. Paul Nation (IBM Quantum).  He is one of the developers of the QuTiP package, another important Python toolbox in quantum physics research.  We cite the QuTiP package in the manuscript as Ref. [7].  His email is paul.nation[at]ibm.com
    \item Dr. Vasil Saroka (University of Rome Tor Vergata).  He is an expert in condensed matter and open quantum systems.  We take some results from one of his papers to be reproduced using \texttt{pyBoLaNO}, for example, Ref. [16].  His email is vasil.saroka[at]uniroma2.it  
\end{enumerate}
    
\end{justify}

\vspace{10px}

\noindent Thank you for your consideration.

\vspace{5px}

\noindent Corresponding Author:\\
\textbf{Hendry M. Lim}\\
Email: hendry01\@ui.ac.id\\
Research Center for Quantum Physics,\\
National Research and Innovation Agency (BRIN),\\
also previously at the University of Indonesia

\end{document}